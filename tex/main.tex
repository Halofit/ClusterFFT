\documentclass[11pt]{article}
\usepackage{geometry}                % See geometry.pdf to learn the layout options. There are lots.
\geometry{a4paper}                   % ... or a4paper or a5paper or ... 
\usepackage{graphicx}
\usepackage{epstopdf}
\usepackage[slovene]{babel}
\usepackage[utf8x]{inputenc}
\DeclareGraphicsRule{.tif}{png}{.png}{`convert #1 `dirname #1`/`basename #1 .tif`.png}
\renewcommand{\baselinestretch}{1.1} % za boljšo berljivost večji razmak
\graphicspath{images/}


\title{Poročilo pri predmetu ZZRS}
\author{Ajda Lampe, Dejan Štepec, Anže Medved}
\date{\today}                                          

\begin{document}
\maketitle

\section{Predstavitev ideje}
V skupini smo se odločili, da izvedemo simulacijo sistema za spletno oglaševanje. Sistem prejme preko HTTP POST zahtevka sporočilo v obliki JSON v katerem je zapisan IP naslov uporabnika, ki si ogleduje spletno stran. Na podlagi IP naslova izvede geolociranje, s katerim pridobi državo uporabnika, ki služi kot ključ za poizvedbo v bazi oglasov. Sistem odgovori na zahtevek s JSON sporočilom, ki vsebuje HTML kodo oglasa.\\

\noindent
Za namene testiranja bomo pripravili strežniško aplikacijo in odjemalca, ki bo pošiljal zahteve na strežnik. Pri testiranju bomo merili odzivni čas sistema, torej čas od poslane zahteve do prejetja HTML kode oglasa. Breme, ki ga bomo pošiljali proti strežniku bo večinoma enolično in majhno, vendar bomo spreminjali frekvenco poslanih zahtev. Breme, ki ga bomo prejemali od strežnika pa bomo po velikosti spreminjali (od nekaj kB do nekaj MB). Za bolj nadzorovano testiranje bodo IP naslovi v odjemalcu določeni vnaprej (naslovi različnih DNS strežnikov), tako da bomo v naprej vedeli, kakšen "oglas" nam mora sistem vrniti.

Na Slika 1 je prikazan diagram uporabe sistema v realnem svetu, na Slika 2 pa diagram simulacijskega okolja.

\begin{figure}
  \centering
    \includegraphics[width=0.71\textwidth]{use_case.png}
  \caption{Diagram uporabe v realnem svetu}
  \label{fig:use_case}
\end{figure}


\begin{figure}
  \centering
    \includegraphics[width=0.85\textwidth]{use_case2.png}
  \caption{Diagram simulacijskega okolja}
  \label{fig:use_case2}
\end{figure}


\section{Izbira ponudnikov oblačnih storitev}
Aplikacijo bomo gostovali pri dveh ponudnikih oblačnih storitev, in sicer DigitalOcean in Amazon. Pri obeh ponudnikih so osnovne storitve za nas na voljo brezplačno - pri DigitalOcean dobimo kot študenti 100\$ kredita, pri Amazonu pa imamo za prvo leto uporabe brezplačnih 750ur delovanja na mesec. Amazonova gruča nam bo omogočala, da bomo lahko hkrati pognali več instanc odjemalcev in s tem povečali frekvenco poizvedb.

\section{Izbira tehnologij}
\subsection{NodeJS}
Kot osnovo za aplikacijo, ki bo tekla na strežniku smo izbrali NodeJS, ki temelji na Googlovem V8 Javascript engine-u. Izbrali smo ga, ker omogoča izdelavo hitrih in skalabilnih omrežnih aplikacij. Zaradi dogodkovno vodenega modela in neblokirajočih vhodno-izhodnih operacij je najbolj primeren za podatkovno intenzivne realnočasovne aplikacije. Velika prednost pa je tudi obsežna zbirka knjižnjic, ki je dostopna preko upravljalca paketov npm.

\subsection{MongoDB}
Bazo bomo implementirali v MongoDB, ki omogoča delo z dokumentno usmerjenimi podatkovnimi bazami. Podatki so shranjeni v bazi v obliki JSON dokumenta, tako da je pridobivanje iz baze z uporabo NodeJS hitro.

\subsection{Python}
Odjemalca bomo napisali v Pythonu, saj omogoča najenostavnejšo pošiljanje HTTP zahtev brez uporabe dodatnih knjižnjic. Ker bo odjemalec napisan v skriptnem jeziku ga ne bo potrebno prevajati, vendar le zagnati.


\section{Implementacija sistema}

\subsection{Strežniška aplikacija}
Pripravili smo strežniško aplikacijo napisano v NodeJS, ki deluje po arhitekutrnem stilu REST storitev. Aplikacija navzven ponuja POST storitev, ki jo kliče odjemalec ko želi uporabiti funkcionalnost strežniške aplikacije.\\

\noindent
Odjemalec ob klicu storitve pošlje HTTP POST zahtevek, ki v zaglavju vsebuje sporočilo v obliki JSON formata. Sporočilo lahko vsebuje več polj, med katerimi mora vsebovati polje myIP v katerem je zapisan IP naslov odjemalca. Strežniška aplikacija spo\-ro\-či\-lo razčleni in pridobi vrednost polja myIP, ki ga nato posreduje knjižnjici za geolociranje.\\

\noindent
Za izvedbo geolociranja smo uporabili odprtokodno knjižnjico geoip-lite, ki preslika dani IP naslov v geolokacijske podatke (država, regija, mesto, ...). Knjižnjica trenunto podpira standarda IPv4 in IPv6, vendar za naslove iz standarda IPv6 vsebuje samo podatke o državi. Več informacij o knjižnjici je na voljo v povezavi vira 7.\\

\noindent
Vrnjene geolokacije podatke aplikacija nato uporabi za izvedbo poizvedbe na podatkovni bazi. V bazo smo za namene prvega testiranja aplikacije dodali samo dva ključa (US,DE) in pripadajoča oglasa v obliki HTMLja. Bazo bomo v naslednjih tednih dopolnili, tako da bo vsebovala večje število ključev, ki bodo vračali različno velike rezultate poizvedb, trenutna ključa, ki sta bila namenjena zgolj za testiranje povezljivosti aplikacije in baze, pa bomo odstranili.\\

\noindent
Rezultat poizvedbe se nato vključi v zaglavje odgovora aplikacije na POST zahtevek in pošlje odjemalcu.\\

\noindent
Aplikacijo smo namestili v oblak ponudnika DigitalOcean, kasneje pa jo bomo namestili še na Amazon AWS.

\subsection{Podatkovna baza}
Na strežniku smo pripravili podatkovno bazo navideznih oglasov, ki bodo služili kot bremena pri testiranju delovanja in zanesljivosti strežnika. Kot je bilo omenjeno v uvodnih poglavjih smo bazo realizirali v nerelacijski bazi (noSQL) MongoDB. Nerelacijsko bazo smo izbrali, ker v primerjavi z relacijsko bazo ponuja večje frekvence poizvedb in manjši čas procesiranja, kar vodi do hitrejšega pridobivanja podatkov.\\
\noindent
Podatki so shranjeni v obliki dokumentov, ki vsebujejo po dve polji - kratico države po standardu ISO-3166-1, ki služi kot ključ za poizvedbo in podatkovnem polju, ki vsebuje navidezno vsebino oglasa. Baza vsebuje oglase petih različnih velikostih, in sicer 25kb, 100kb, 300kb, 520kb in 1000kb. V primeru, da se bo pri testiranju pojavila potreba po obsežnejših bremenih, bomo bazo dopolnili.

\subsection{Odjemalec}
Napisali smo odjemalca v jeziku Python, ki na strežnik pošilja POST zahteve in meri čas od poslane zahteve do konca prenosa odgovora. Pri delovanju izpiše čas posamezne zahteve, po koncu pa še skupni čas, povprečni čas, maksimalni čas in minimalni čas.


\section{Plan za naslednji teden}
V prihajajočem tednu imamo namen dopolniti odjemalca tako, da bo podatke shranil v datoteke, ki bodo enostavne za obdelavo in računanje statistike, saj smo se trenutno usmerili bolj v človek razumljiv izpis za namene testiranja. Pričeli pa bomo tudi z vzpostavitvijo Amazonove gruče in se tako pripravili na izvajanje meritev.

\section{Viri}
\begin{enumerate}
  \item https://www.digitalocean.com/
  \item http://aws.amazon.com/
  \item https://nodejs.org/
  \item https://www.npmjs.com/
  \item https://code.google.com/p/v8/
  \item https://www.mongodb.org/
  \item https://www.npmjs.com/package/geoip-lite
\end{enumerate}

\end{document}
